\section{About Workflow}

This fully computer-generated image touches upon many modern techniques in color management, including a scene-referred linear approach to rendering, shading, and illumination, in addition to on-set lighting reconstruction and texture management.
Visual effects by Sony Pictures Imageworks. Images from “The Amazing Spider-Man” Courtesy of Columbia Pictures. © 2012 Columbia Pictures Industries, Inc. All rights reserved.

This section will describe techniques, choices and considerations that go into modern color workflows for live-action, visual effects, animation and games productions. The section starts with the introduction of two subjects that will be of use throughout the rest of the chapter: the Academy Color Encoding System and the process of crafting and managing the Look of a production. The rest of the chapter will follow an order roughly driven by the chronological order of work on a live-action visual effects movie, starting with On-Set camera, Lighting and Reference Capture, moving on to the phases of the computer graphics pipeline used in visual effects, animation and games including Texture Painting, Matte Painting, Lighting, Rendering and Compositing and finishing with Grading, Monitoring and Mastering.

To give the discussion context, a Color Pipeline will be defined as the set of all color transformations and encodings used during a motion-picture production.

A well-defined computer-graphics color pipeline rigorously manages the color transformations applied at each stage of the process, in addition to the transforms required for accurate image preview.
In a properly color-managed pipeline, the color encodings for all image inputs, image outputs, intermediate representations, displays and exports are carefully tracked. The transforms used to convert between color encodings are also rigorously defined and tracked. A well-managed pipeline protects the maximum color data available and uses the fewest transforms possible.
