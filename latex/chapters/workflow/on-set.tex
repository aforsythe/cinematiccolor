\section{On Set}
\label{sec:on-set}

An illustration of the recommended color managed approach for cinema or TV
*This LUT is generated from a CDL or proprietary saved grade format, which is then updated with any changes made in the dailies grade, and passed to DI as a start point


During filming, the aim is to capture the greatest amount of information possible, at the highest quality possible in order to give maximum flexibility in post-production, and to facilitate the integration of VFX elements. From a camera perspective, this means shooting in a raw format or log encoding and exposing to ensure that as much as possible of the dynamic range of the scene is captured. The way modern digital cameras apply the Exposure Index (EI) set while shooting is to alter the sensor level which maps to mid-grey, and in doing so a trade-off is created between the amount of latitude available above and below grey. The total dynamic range of a sensor is fixed. Rating a camera at a lower EI will mean lighting the scene more brightly for the same exposure, resulting in lower noise, which is beneficial for VFX. However, because doing this also reduces highlight latitude, care should be taken that important highlight detail is not clipped. In a studio, where the lighting can be controlled, this is not normally an issue, but on location, it is important to be aware of this limitation.

The monitoring output from the camera is the first opportunity for the production to get a preview of the intended look of the motion picture. When a look has been designed during pre-production, this can be applied to the signal sent to on-set monitors, complementing that part of the aesthetic of the film which comes from production design and costume. The look of a film can easily become ingrained in people’s minds based on how they first see it, so taking the opportunity to present the image in at least an approximation of its intended form at the earliest possible stage can be very beneficial. However, care should be taken if using extreme looks that the appearance of the image transformed through them does not negatively impact exposure decisions. It is advisable to also check the image through a standard neutral display transform, to ensure that a well exposed image is being recorded, clear of both the sensor noise floor and highlight clipping, and to allow for the possibility of different decisions about the look being made in DI grading.



\subsection{Camera profiles}
\label{subsec:camera-profiles}

Color management should be used at every stage of the pipeline, and for material acquired with a digital camera, this can begin on set. All current digital cinema cameras capture a wide dynamic range and color gamut, and their encoding primaries and transfer functions are published by the manufacturers. Whether the recording is being made in a raw or camera log format, a log signal is usually available from a monitoring output of the camera. These are high-dynamic range, scene-referred representations of the image and can be used to apply a viewing transform, with or without a look included, which can be replicated and refined as the image passes through the various stages of post-production.


As can be seen from the above plots of some common camera log curves, the basic shape of all the curves is quite similar, with each having a straight line for a large part of the curve with a toe where it is compressed towards black. The value for lens cap black is often based on the 10 bit Cineon code value of 95 for Dmin. The straight line segment, in a plot with a logarithmic x-axis such as this, shows that the encoding is truly logarithmic in that part of the range, with an equal number of code values per stop. Its slope varies between encodings, as does the exposure represented by a log value of 1.0. It is therefore important to know what the encoding curve used for a given log image is, in order to be able to linearize it correctly. A LUT or ACES Input Transform designed for one camera may give a superficially pleasing result with an image from a different camera  but could cause unwanted clipping or distortion which is not obvious on all images. Using an incorrect LUT or Input Transform for aesthetic purposes is a risky strategy, which can cause problems further down the line. It is strongly recommended to use a correct linearization, and a separate operation for the look. It is important to understand that although a camera may use an encoding curve which has a particular dynamic range, this should not be taken to suggest that the camera itself can capture that range. The Sony FS7 for example, at default EI, clips six stops above mid-grey which maps to an S-Log3 value of 0.866. Values above this will not occur, even though S-Log3 can encode 7.74 stops above grey. Details of the encoding and decoding equations for various log curves is given in Appendix 4.1.

The cameral log encoding of the ‘A’ Camera on a production is often chosen as the working format for the final DI, since log encodings lend themselves well to the kinds of adjustments made during grading. They are constrained to the 0-1 range, or integer encoding of that range, and map the dynamic range of the scene evenly across the code values. Images from other types of camera can be transformed into the ‘A’ Camera’s log encoding. Alternatively, all cameras can be transformed into a non-camera-specific log working space such as ACEScct for grading. Ideally, this choice should be made in advance in consultation with the DI colorist, so any grading adjustments use the same working space.


Plots of the primaries of some common camera encodings

The above diagrams show the encoding primaries used by various digital cinema cameras, as well as those of ACES2065-1 and ACEScg. As can be seen, the CIE xy coordinates of the primaries vary considerably, so it is clear that a transform or LUT designed for one, will not work as intended on another. It is important also to be aware that these are simply encoding primaries, to which the raw sensor output of a camera is transformed. While they define the gamut of the encoding color space, this does not mean that the camera system in question can ‘see’ every color within that gamut. Indeed, since some primaries have negative coordinates, the encoding can represent colors which have no meaning in CIExy encoding. Despite this, a camera may produce pixels which appear to have these colors, particularly when lights with discontinuous spectra are visible in the frame. This is due to the residual error in the matrices chosen by the manufacturer to transform from the sensor’s raw RGB to the encoding color space. Cameras are not truly colorimetric, and so no 3×3 matrix can produce a perfect result. Manufacturers choose a matrix which minimizes errors in important colors such as skin tones. Care must be taken with images which contain these unreal colors, as the negative pixel values they contain may cause unwanted artifacts.

A more accurate matrix could be derived specific to the precise shooting setup, lighting, lenses, and filtration, but unless it is vital to have as photometrically accurate a result as possible, it is normally preferable to use the manufacturer provided matrix, such as those in the published ACES Input Transforms, to ensure matching processing by all vendors handling the media.


ARRI ALEXA, LogC, ALEXA Wide Gamut	ACES Rec. 709 Output Transform (pushed 0.5 stops)

Canon C500, C-Log3, Cinema Gamut	ACES Rec. 709 Output Transform (pushed 0.5 stops)

RED Dragon, Log3G10, RED Wide Gamut	ACES Rec. 709 Output Transform (pushed 0.5 stops)

Panasonic Varicam PL, V-Log, V-Gamut	ACES Rec. 709 Output Transform (pushed 0.5 stops)

Sony F-65, S-Log3, S-Gamut3.Cine	ACES Rec. 709 Output Transform (pushed 0.5 stops)
Images courtesy of Geoff Boyle, cinematography.net

The images above show how footage from different cameras can be matched by bringing them into a common working space and then applying the same display transform. The camera log images have been normalized by applying gain in scene-referred linear to bring the white chip of the chart to RGB = [0.9, 0.9, 0.9]. The difference between the log encoding curves is apparent as the differences in percieved contrast between the various log images, even though they encode the same scene luminance. The difference in the location of the encoding primaries manifests as variation in the apparent saturation of the log images, noticeable particularly in the green part of the chart from the RED camera. These differences in the way the same scene colorimetry is encoded are removed by using the appropriate Input Transform to convert to the ACES2065-1 colour space. The display transform is then applied by adding a half stop push for aesthetic reasons, and then using the ACES Rec. 709 Output Transform. It can be seen that even with the manufacturers’ published Input Transforms, there is still variation, particularly in the rendering of skin tones. The difference in the spectral responses of the different sensors (in fact primarily the differences between the spectral characteristics of the Bayer color filters on the sensors as well as the IR filtration) and the fact that some artistic choice is involved in characterizing a sensor, mean that each camera has slightly different feel to its image; this is unrelated to the differences between the appearance of the log encodings. ACES Input Transforms will not make all cameras look identical, but should provide a reasonable match up within the limits of each camera.
XXX
Key Points
Cameras from different manufacturers use different log encoding curves and primaries to represent the same scene colorimetry
These are just encoding methods, and if properly interpreted have no impact on the final look of the image
Cameras still have varying characteristics which do affect the look of the image
ACES Input Transforms bring different sources into a common working space so that the same Output Transform can be used for all sources, giving a close match of appearance

\subsection{Monitoring on set}
\label{subsec:monitoring-on-set}

Most of the monitors used on set are likely to be provided by the video department and, since they are intended for checking of performance, framing and focus, may well not be accurately color calibrated. In order to maintain consistent color, at least one monitor should be properly calibrated, and in a more suitable environment (as specified by ITU-R BT.2035) for critical viewing of images, such as a blacked out tent. This is usually the responsibility of the Digital Imaging Technician (DIT).

The image displayed on this monitor is derived from the log feed from the camera, allowing the DIT to examine the unmodified scene-referred image using a waveform, false color, and other tools, in order to check exposure and shadow and highlight latitude. The image can then be viewed through the desired display transform, applied either by a LUT box or internal processing in the monitor.

The log output from the camera is likely to be a 10 bit SDI signal (often mirroring the floating point log encoding used in the DI) and is therefore well suited to the application of look transforms to preview the intended appearance of the film. However, it may well not represent the image at the full precision being recorded. Therefore, rather than being baked into the image, looks are normally represented in metadata form, which can travel with the rushes to be applied as appropriate later. The transform applied may be one universal Show LUT, per scene LUTs, the default LUT for that camera, or the image may be graded live by the DIT using dedicated software.

The ACES system defines an XML sidecar file, called ACESclip, which carries various metadata, including Look Transforms. At the moment this is not widely supported by post-production software, so while it may well become the de-facto approach in future, other workflows are normally used at this time. This can consist of proprietary saved grade formats or project files from DI software, or CDL may be used for greater compatibility at the expense of sophistication. See Appendix 4.5 for a description of the CDL format. LUTs are usually also created for baking into dailies and other look preview purposes.

A dailies colorist may then adjust the transform on a per shot basis to balance the image. Whatever transform is baked into the dailies and editorial, transcodes must then be tracked through post-production, so that VFX artists can see the same look when working on a shot, and bake it into renders for editorial, thus matching the original plate. If the colorist updates the grade, in advance of the final color, the transforms used in VFX should be updated to match this.

Something to be aware of when dealing with log encodings is that although, for example, both LogC and S-Log3 code black as ~0.093, the output of an ALEXA and an F55 with the lens cap on will be different. The ALEXA output will be at 9.3% on a waveform, and that of the F55 will be at 3.5%. This is because Sony Define S-Log3 in relation to the unscaled SDI code values, whereas ARRI define LogC in relation to ‘IRE’, represented by the 64-940 SMPTE range over SDI. See Appendix 4.7 for more detail on these Full and Legal ranges. This needs to be taken into account when building LUTs for use on set, as does the processing of the LUT box used, which may apply the LUT to the unscaled SDI code values, or may map 64-940 to float 0-1 before applying the LUT, and then back to 64-940 afterward.

While wide gamut and HDR monitors are available for use on set, it is still most common for the DIT’s monitor to be calibrated to the BT.1886 / Rec. 709 standard. Since there is a great deal of creative choice available in an HDR grade, and indeed variation in HDR view transforms, viewing the image with a generic HDR output transform will not necessarily be representative of the final HDR deliverable. Although the camera may be capturing a wide color gamut, the vast majority of colors found in photographic images are contained within the Rec. 709 gamut, so using this as a baseline output transform is normally considered adequate. Where HDR monitoring is used on set, an SDR version should always be referred to as well.

Key Points
The log output from a camera can be used to apply a look using the same approaches as will be used in the DI
Looks exist in the monitoring path only, and are communicated as metadata
ASC CDL is the lowest common denominator for communicating a look, as it operates identically in all systems


\subsection{Editorial}
\label{subsec:editorial}

Although modern Non-Linear Editors (NLEs) are capable of applying LUTs and other display transforms in real-time during playback, there is always a system overhead for doing this. In editorial, the ability to scrub quickly through the rushes is crucial, and anything which slows this process down can be problematic. Therefore the usual practice is to create transcodes in the native format of the NLE (either Avid DNxHD or Apple ProRes) with the look and output transform baked in. This is done either on or near set by the DIT or by a dedicated dailies colorist in a lab.

Metadata for the rushes is passed to editorial in the form of Avid ALEs or database files, and contains information about the display transform applied, as well as CDL grades and other information from set. This can include notes from the script supervisor, and information on lenses and filters used. Accurate tracking of this metadata through editorial is vital to ensure correct handling of the material further down the line. For VFX shots, a representative from the vendor is often present on set, and their notes can be cross-checked later with the metadata, to ensure the choice of correct lens grids and so forth.

Key Points
Baking the look into dailies for editorial is current best practice for practical reasons
Tracking look and other metadata through editorial is vital

\subsection{Digital Imaging Technician (DIT)}
\label{subsec:digital-imaging-technician-dit}

It is the Digital Imaging Technician’s (DIT) responsibility to liaise between set and post-production. They need to communicate with those who will be responsible for dealing with the image further down the line, as well as with the cinematographer, to ensure they have an understanding of the intended color pipeline and required deliverables. The role originated as the person responsible for understanding the controls in video cameras, as productions moved to HD video from film acquisition so that the cinematographer could focus on lighting and exposing as they had done in the past.

Historically live grades were often applied to an output-referred signal from the camera, with a base viewing transform already baked in. The same grade applied before or after a display transform will give a very different result. This limits the look modifications that the DIT can apply to crude approximations of the intended look, useful only for visual reference rather than as reusable metadata. This is comparable to the way cinematographers have in the past experimented with looks in Photoshop. Now tools are becoming available which allow the cinematographer to craft a look with scene-referred images, using the same tools as will be used by the DIT on set, and later in the DI. The DIT should work with the cinematographer to ensure that this process is understood, and works smoothly.

If a live grading system is used, the DIT is responsible for creating primary grades to improve consistency from shot to shot. They will often maintain a library of stills from previous setups or locations, to aid the cinematographer in matching the lighting, and to even out the differences still further using simple color corrections. It is important if grades created on set are to be successfully passed on and used later, that the working space in which the color operations are applied, and the view transform used, is agreed beforehand and communicated clearly so that the look can be replicated accurately. Even within the standardized ACES pipeline, there is the option to grade in ACEScc or ACEScct. The result of using either of these is identical in the mid-tones and highlights of the image, but the look of the shadows will be different.

Key Points
A DIT must communicate with the camera department and post, and liaise between the two
For live grading to be truly useful it must use the same color management approach as will be used further down the line

\subsection{Reference Capture}
\label{subsec:reference-capture}


Typical standalone color checker rig for use in the VFX industry. The image has been white-balanced using gain adjustment on the Neutral 5 (.70 D) swatch, i.e. third swatch from the right of the bottom row, so that its value matches the reference sRGB luminance of 0.192. Note that while this is slightly higher than the 18% of a grey card, it usually does not matter in practical applications, and is commonly set to 0.18. If fine precision is required, it is important to keep that difference in mind. Also note that the grey ball in this particular image exhibits excessive glossiness from years of usage and should be recoated.

Along with the main camera image, it is important for VFX shots to capture additional reference, data, and metadata about the scene being photographed. Reference images of the set or props should be paired with images of other references such as a Macbeth Chart and an 18% grey card. Ideally, the reference images and reference objects would be captured using the main shooting camera and a DSLR that can store RAW images. If possible, it is best to use conversion software such as DCRaw or OpenImageIO to produce accurate scene-referred data. The Macbeth chart and 18% grey card can be used to balance the exposure of the reference and to remove the gross effects of the lighting environment if the reference will be used for texturing. If the reference capture is to be used for texturing specifically, it is best to shoot objects in a separate, as diffused as possible lighting environment. Professional photogrammetry services use white rooms with lights bounced off the walls or through diffusers, to achieve this effect.

Depending on the complexity of the VFX work envisaged, it may be necessary to take LIDAR scans of the set or HDRI environment captures. Chrome and grey spheres are often shot as quick lighting reference but should not be relied upon beyond that. The color handling of the LIDAR scanner or HDR capture device should be characterized before being used on set so as to understand the limitations of the devices. Most LIDAR scanners don’t capture color at either sufficient resolution or bit depth, so the result shouldn’t be used for anything beyond alignment. Many turnkey HDR capture devices, for example, offer only limited specification or control over color gamut. They mainly focus on the intensity range that can be captured. In this case, it is essential to shoot a Macbeth Chart with the main shooting camera, a DSLR used for reference and the HDR capture camera.

Lens distortion grids should be shot with all lenses used on the production, and notes should be taken as to which lens is used on each shot. These will be used later to remove lens distortions and vignetting.

For VFX shots it is inadvisable to simply rely on the camera sheets or script notes, as these can often conflict. Embedded metadata in the captured images can be invaluable in helping verify notes and report data.

\subsection{HDR Environment Capture}
\label{subsec:hdr-environment-capture}

An HDR environment map captured for ‘The Avengers: Infinity War”. Luminance ranges go from 0 to 17100
Images are © 2018 MARVEL

Panoramic HDR captures are the standard approach to capture onset lighting. A new HDR panorama should be captured each time the lighting changes. The classic technique to capture HDR panoramas for use as environment maps, first pioneered by Gene Miller, was to photograph a chrome ball, as  Recent advances in hardware now allow for directly capturing the scene-illumination, either with a camera using multiple, bracketed exposures or with dedicated hardware. One extension to this methodology is to capture the scene from multiple locations or heights. Triangulating the location of the lights in the series of captures allows for placement of area lights in the 3D scene with plausible energy estimations. Placed area lights allow for physically plausible rendering of local illumination effects, which contrasts with traditional skydome rendering approaches which assume infinitely distant lights. See Duiker (2003) and Selan (2011) for additional discussion.

Care must be taken when calibrating HDR lighting captures to account for colorimetry, linearity, and white balance, or the resulting lighting data may not integrate well with the computer-generated environments. It is common to use ND filters to darken the exposures, capturing even more information for the brightest parts of the environment for a given shutter speed and aperture. Fisheye lens usually won’t work well with ND filters. Adding a diffuse sphere and other reference materials during scene capture is useful in validating the reconstruction of the lighting later on. For outdoor day-lit environments, reference photos of a diffuse sphere may be used to recover or validate the sun’s light intensity, as it is not captured properly by even the best cameras.

There are some best practices when choosing the exposure and white balance for a HDR lighting capture, but there is an element of subjectivity in the choice of approach. Establishing the exposure of an 18% grey card facing the direction that you would expect an actor or the subject in the environment to face is one valid approach. Another valid approach is to establish the exposure of the 18% grey card when facing each of the cardinal directions, then to use the average of those exposures as the exposure for the scene. The white balance is often established using similar approaches. An additional approach to establishing exposure and white balance is to leave a grey card or Macbeth chart on the ground in the scene and then to neutralize the grey card or grey chips from the chart. This has the advantage of integrating lighting from the entire upper hemisphere. Note that much like primary color correction, the grey card should be neutralized only using simple multiplies with scene-referred linear imagery. Using matrices, adds, saturation effects or other more complicated operations could compromise the linearity and dynamic range present in the panorama. See Lagarde (2016) for a detailed discussion of the techniques and choices involved in capturing high quality HDR panoramas.

