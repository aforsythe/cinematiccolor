\section{The Look}

The visual style or Look of a production is integral to the audience’s experience of the media, be it live-action, animation or game. It evolves from pre-production through to delivery and can be modified anywhere in the pipeline. In live-action, the Look is often adjusted, finessed and approved in the DI grade, which is the final stage of post-production. In animation and games, the overall process is very similar except that the finished Look is established earlier on during content development and a final DI grade is less common or less impactful to the Look.

The Look is the product of design, lighting, and grading. Crafting the Look involves many collaborators, many ideas and many processes. In order to view camera and CG scene-referred content in a meaningful way, it is important to add an approximation of the Look and an output transform that matches the display in use. Look Management is the process of accurately communicating the intended Look throughout production.

The existence of a look should always be considered even if it is not explicitly defined in a color pipeline. The earlier the Look is established the more decisions on lighting, VFX and editing are based on a common understanding. Even without making a choice, the project still has a look.

\subsection{Establishing the Look}

Design influences in pre-production begin with the producers and director discussing ideas. The Look begins with these ideas and concepts, sometimes collected in the form of a design scrapbook. In live-action, the Directory of Photography (DP), costume and set designers, makeup, the post-production supervisor, concept artists, and others then work together to develop an early version of the Look, which is then built into a look transform as a LUT, ASC Color Decision List (CDL) or other similar proprietary formats. These early discussions drive the selection of cameras, lighting, stages, locations, costumes and image processing. Sometimes extensive camera and DI grading tests are made to confirm viability or assist in the decision-making process, but usually, ideas remain relatively fluid in the early period. The director's scrapbook often helps to communicate the influences as well as the intent.








A selection of final images from live-action visual effects, animation and games. The look of each is driven by the creative direction of the production. In this small selection, you can see a variety of choices around contrast, saturation and dominant colors.
Images are © 2018 MARVEL, © Disney/Pixar,  © Disney 2018 and © and ™ Lucasfilm Ltd. All Rights Reserved. Imagery from Battlefield V courtesy of Electronic Arts Inc, © 2018 Electronic Arts Inc. All rights reserved. Images © Disney/Pixar © Geoff Boyle All rights reserved.

If the Look evolves progressively from an established concept, through Look Management, and into the DI grade it results in higher quality deliverables and can also save a great deal of time, effort and money. Even with the most careful Look Management, there are likely to be many modifications in the final stage, the DI grade. More of this is discussed later in the 3.6 DI Grading. Sometimes the colorist may be asked to dramatically diverge from the original artistic intent, and this may be for any number of reasons.





A shot from “Avengers: Infinity War” with a generic Rec. 709 look (top) and the show’s actual look (bottom)
© 2018 MARVEL
Once source images exist it is up to Look Management to enable all the contributors to view the images as intended, then for DI grading to perfect the Look on a scene by scene basis and to manage elements within each frame if necessary. The DI colorist is also responsible for managing any differences that occur as a result of different color gamut, dynamic range or resolution in the final delivery versions. There are two schools of thought on this. Some studios prefer that all versions are as similar as possible and are prepared to surrender the extremes of gamut and dynamic range to ensure that. Others seek to exploit the full use of high-dynamic range and wide color gamuts even though the differences between delivery versions are more apparent.

Crafting the Look in the widest color space, highest dynamic range and highest resolution provides a more future proof master deliverable, especially if the project remains scene-referred. Some organisations use this as a marketing differentiator. Lighting, makeup and production design decisions for an HDR screen master are very different to those made for an SDR television master, so it is best to consider the primary target display technology as early as possible.



A shot from “Rogue One: A Star Wars Story” with a generic Rec. 709 look (top) and the show’s actual look (bottom). The source colors for the bright magenta / pink of the X-Wing’s engines were outside of the Rec. 709 gamut.
© and ™ Lucasfilm Ltd. All Rights Reserved
In some productions, particularly in animation, the Look is set in early pre-production, with a little latitude for DI grading allowed to help continuity and atmosphere. There are situations when the Look may be quite different for some deliverables, e.g. for HDR Displays, or the Look may evolve over time, for example, TV series can look different between early seasons. These may need to be better matched for box sets later on. Often, the Look is not fully defined until the DI grading stage and then the extra flexibility of source referred VFX elements is advantageous. The Look can change even if the final DI colorist is involved from the very start either on-set or for the dailies. Changes to contrast, color or brightness might behave differently on CG material compared to camera sourced media. Whilst it is the intent in an ideal world that CG elements exactly match the scene-referred pixel values that the live-action camera would have captured, in reality, the capture process itself may have variations. Then the colorist has to correct for continuity as well as apply the Look. Differences between elements are also likely to occur if the Look Management is not applied correctly across all departments and facilities.



 A live-action image from a camera test shoot with an ACES Rec. 709 output transform (top) and a test grade crafted to even out the skin tones (bottom).
Images are © Geoff Boyle
The responsibility of approving the Look can fall to the producer, director, cinematographer, colorist, VFX supervisor. Sometimes even the editor is involved. Often there are disagreements on the Look and the way it is interpreted scene-to-scene. For that reason, it helps to clarify whose vision the post-production team are working towards. Without clear direction, much time will be lost preparing alternate versions or re-rendering fixes in the final DI grade as decisions change. Whoever makes the decisions on the Look, someone then has to take responsibility for it through to delivery. That is usually either the Director of Photography or the colorist. The senior colorist is likely to be more accessible but may not be appointed until much later in the post-production timetable. Whilst there is a strong tendency for everyone to become familiar and comfortable with the Look set in the color managed pipeline it can happen that quite significant changes are made at the final grading stages in order to better communicate an impression to the audience.  Often the full resolution images are not seen on a large calibrated display in the edited sequence until the DI master grade when there is no longer an obligation to use reversible transforms. The DI colorist  can then finesse each scene to work within the Look or tweak the Look to better convey the artistic intent.

The Look for animation is guided by artwork created in visual development. Translating the creative style into a Look that can be applied to CG animation is an iterative process utilizing render tests of neutrally lit character rigs as well as more complete scenes that are representative of the storytelling environment. In a facility where there is a history of previous Looks, one of these may be selected as a starting point for further development. It can also help to start with a relatively neutral look before incorporating specific creative touches. A neutral look is useful for verifying the technical quality of renders and the accuracy of physical simulations.




A progression of Looks. (Top) A generic gamma 2.2 output. (Middle) Standard show Look with contrast and saturation boost, plus highlight compression. (Bottom) Final stylized Look incorporating orange-yellow filter.
 Images are ©Disney. All Rights Reserved.

The initial pass at the Look tries to establish the desired contrast and overall shape of the tone curve along with the appropriate level of color saturation. Once this is established more specific adjustments can be made to dial in characteristics such as the behavior of bright highlights and shadow detail. A controlled amount of color distortion may be desirable sometimes, for example, to lend warmth to highlights. These fine adjustments may not be easy to make with a general test image, but can be made more obvious by selecting the right image content. Characters are particularly useful for fine-tuning the appearance of facial contours, highlights in the hair and the overall reproduction of skin tones, while environments are useful for evaluating color relationships and setting the behavior of very bright light sources. Because the final viewed color is a product of both the rendered or captured image and the Look it is viewed through, it is important to set the Look early so that changes do not impact approved images once production is underway. Once a base Look has been set for the movie, it may also be complemented by secondary Looks to support the story as it shifts into different environments.




A generic Rec. 709 look (top) and the project’s actual look for this scene (bottom).
Imagery from Battlefield V courtesy of Electronic Arts Inc, © 2018 Electronic Arts Inc. All rights reserved.
The Look for a games project is established in much the same way as for an animated feature. Tests using environments and characters are run to establish the desired contrast, saturation and overall shape of any S-curve that may be required (if any). In the context of a game, a Look has to serve not just the vision of the art director but also the goals of the gameplay designers. If a Look is overly contrasty, it may interfere with gameplay by making enemies hiding in shadows effectively invisible to the player. By the same token, if the tone curve’s shoulder is too abrupt, the detail in the highlights may be lost, which can hinder gameplay if bright objects or enemies can’t be differentiated from bright skies or lights. Content is often validated against the game Look as well as a neutral Look. Since games are real-time and often built as a result of rapid iteration, the ability to change is held in high regard and thus content may be reused in multiple very different Looks. For example, gameplay may require a different Look for low health, or the game may be played at different times of day. Thus validating the content holds up in both the target Look and a neutral Look gives confidence the lighting or Look can be changed without requiring significant asset rework. Checking content in a neutral and project-specific Look mirrors animation and live-action workflows.

Games can be made with fairly small teams and as such, one person may perform many separate roles. There is generally no one person who tends to own Look creation, though the art director will typically drive it and make the final call to sign off on assets, lighting and Look adjustments made by a team of artists. One difference with live-action and animation is that because game engines generate new images every time the game is played, the opportunity to change and refine the Look extends even closer to the time when the audience sees the finished frames even tuning it to suit the individual display being used. By association, they have less ability to fall back to manually created and fine-tuned trim passes like visual effects and animation projects to optimize the display quality for different displays’ dynamic range and gamut. Although broad operating parameters can be set for the engine to work within and interpolate accordingly. Efforts such as the “HDR Gaming Interest Group” (HGIG) aim to optimize the images generated to best utilize the characteristics of each display.

Key Points
Even without making a choice, each project has its Look.
The Look influences audience perception of imagery.
The earlier the Look is established the more effective it is implemented.
Testing with a neutral look is useful for verifying technical quality and accuracy
The base Look may be complemented by secondary looks for some scenes


\subsubsection{Displaying Scene-Referred Imagery}

One of the complexities associated with scene-referred imagery is display reproduction. While scene-referred imagery is natural for processing, displays can only reproduce a much lower dynamic (LDR) range. While at first glance it seems like a reasonable approach to directly map scene-referred linear directly to the display, at least for the overlapping portions of the contrast range, in practice this yields unpleasing results. See “Color Management for Digital Cinema” Giorgianni (2005), for further justification, and Section 3.2 for a visual example.

This challenge has been approached and intuitively solved by artists for millena. The world is a high dynamic range environment. Artists have striven to render artistic versions of reality on low dynamic  range media.  In the color community, there are many aspects to pleasingly reproducing high dynamic range pixels on a low dynamic range display, but one of the most important is known as tone mapping and is an active area of research. Surprisingly, many pleasing tonal renditions of high-dynamic range data to low dynamic range displays use similarly shaped transforms. On first glance this may be surprising, but when one sits down and designs a tone rendering transform there are convergent processes at work corresponding to what yields a pleasing image appearance. Other considerations for reproducing high dynamic range on low dynamic range displays include gamut mapping, highlight bleaching and saturation changes.

First, most tone renderings map a traditional scene grey exposure to a central value on the output display. Directly mapping the remaining scene-referred linear image to display results in an image with low apparent contrast as a consequence of the display surround. Thus, one adds a reconstruction slope greater than 1:1 to bump the mid-tone contrast.  Of course, with this increase, in contrast, the shadows and highlights are severely clipped, so a roll-off in contrast of lower than 1:1 is applied on both the high and low ends to allow for highlight and shadow detail to have smooth transitions. With this high contrast portion in the middle and low contrast portions at the extrema, the final curve resembles an “S” shape as shown below.

The ACES sRGB Output Transform applied to a ramp in the ACEScc color space.
An “S-shaped” curve is traditionally used to cinematically tone render scene-referred HDR colorimetry into an image suitable for output on a low dynamic range display. The input axis is an ACEScc logarithmic base 2 coding of scene-referred linear pixels. The output axis corresponds to code-values for an sRGB display.

The final transfer curve from scene-referred linear to display is shockingly consistent between technologies, with both digital and film imaging pipelines having roughly matched end to end transforms. If a film color process is characterized from negative to print, it almost exactly produces this “S-shaped” transfer curve, which is not surprising given the pleasant tonal reproductions film offers. In traditional film imaging process, the negative stock captures a wide dynamic range and the print stock imparts a very pleasing tone mapping for reproduction on limited dynamic range devices (the theatrical print “positive”). Broadly speaking, film negatives encode an HDR scene-referred image, and the print embodies an output-referred tone mapping. For those interested in further details on the high-dynamic range imaging processes implicit in the film development process, see the Ansel Adams Photography Series [Adams 84].

It should be noted that while there is significant convergence on filmic tone mapping, a filmic curve may not be desirable in games, tv or film. Productions choosing to reproduce a particularly stylized visual, such as a cartoony or highly ambient-lit appearance, may not wish to see the additional contrast or toe from such an operator, so may elect to have a less pronounced curve.

To summarize, it is strongly advised not to directly map high-dynamic range scene-referred data to the display. A tone rendering is required, and there is great historical precedence for using a global “S-Shaped” operator. HDR scene-acquisition accurately records a wide range of scene-luminance values, and with a proper tone mapping operator, much of this range can be visually preserved during reproduction on low-dynamic range devices.

Please refer to Section 3.2 for a visual comparison between an S-shaped tone mapping operator versus a simple gamma transform.

Specific considerations taken into account when crafting the Look for a film, animation or game are discussed later in 3.6 DI Grading

\subsection{Look Management}

Locking down the Look early on is a critical first step in crafting cinematic images, as it underpins every other part of the process. The remainder of the pipeline relies on visualizations that are created by working, translating or transcoding the Look and this is known as Look Management.

An overview of the filmmaking process shows that Look Management is core to each stage. The mistaken belief that the Look is the product of just the DP during shooting, or the colorist during mastering discourages a well-managed pipeline, without which the visual effects process is rather in the dark. Decisions earlier in the chain cannot easily be protected without a carefully color managed pipeline. Meaningful tests should be performed early on, saving a lot of work and some potentially unpleasant surprises later. In high-end productions changes to the Look can be suggested all through post-production and communicated to everyone else via color decision lists of some sort. The CDL is one such mechanism but is quite limited in scope, there are other more complex proprietary solutions in use.

When a look is applied to imagery as it is captured or saved to disk, it’s called “baking the Look”. The Look should not be baked in at the source or early in post-production. If the Look is baked in early, all the problems of output-referred workflows are very apparent, especially if the Look is strong. Continuity is more difficult and there is very little latitude to alter or modify the Look later on. Exposures, color balance, and lighting have to be close to perfect since there is little more information for the colorist to use in the final stages. VFX could still be delivered scene-referred, but the expectation is that they match the main content. Delivering to multiple formats is still possible, but usually, any differences that take advantage of wider tonal range or color gamut are severely limited. It is strongly recommended that the Look is NOT baked in until the final DI grading stage.

Working with multiple gamuts is a subject that can’t be avoided for a show of even moderate complexity. Almost every camera has a different encoding gamut. The working gamut used for a show will likely be distinct from any of the camera encoding gamuts. The working gamut will also likely be distinct from the final output display gamut. This is guaranteed if the show will be seen on multiple types of displays. The advent of the multi-gamut workflow has added an additional layer of complexity to setting up a project and communicating between artists, departments and companies. As such, it is especially important to define the choice of gamuts to be used for capture, grading, lighting, rendering, compositing, and final display as early as possible and to agree on how and when the gamut choices should be presented to users. For many artists, a successful setup will be one where they aren’t aware of the gamuts being used. OpenColorIO (2003)’s “role” feature allows some of these choices to be safely hidden from artists. It is very easy to get lost in a thicket of similar, but not quite matching, gamuts if images and color data aren’t properly tracked or labeled.

One of the benefits of ACES in the world of multiple gamuts is that it provides a clear standard for mapping between the different sets of gamuts used for cameras, rendering, compositing, grading, and final output.

The choice of working gamut affects the feel of controls and operations that affect color. Choosing a space that is too wide may break some operators, like keyers, or may just make otherwise familiar operators and controls feel unfamiliar to artists. Working in the camera encoding gamut may be useful for these jobs. For some operations, a compositor may transform to the camera encoding gamut, apply the operation and then transform back to the working gamut. This is similar in spirit to the time-honored trick of resizing in log described in 3.5.4 Working with Log Imagery.

When the reference Look is created in a DI grading system, the process is saved just as it is in case it needs to be modified or used to derive other looks. The Look itself is sampled in a 3D LUT format that can be loaded into color managed applications. Because some applications have specific requirements for the LUT formatting, it is not unusual to generate multiple LUTs. However, it is preferable to find a common LUT format that works with as many applications as possible with a high fidelity. Exporting the Look as a LUT is the easiest way to share it but using a parameter based method such as CDL is also an option for simpler looks. Many facilities use the combination of a show LUT with a CDL in front of it.

Once the Look has been established for a CG animation project the process that created it is saved just as it is in DI grading systems. The best solution is to use a color management system, like OpenColorIO, that is shared across all the applications and can apply Looks contextually. This allows for a high degree of automation in Look Management and can help eliminate operator error. Tagging images with metadata can assist this system by automatically managing looks across productions and also within a production when multiple looks are employed. A good color management system should be able to generate notes that can feedback to artists working at their desktops and also feed forward to the colorist for DI grading.

Games tend to be produced primarily using a single toolset, the “engine”. Within the engine itself, color management is largely handled automatically, being a single piece of software. For assets that are created externally to the engine, like textures and sometimes materials, it is desirable to have a representation of the Look of the project that can be used in the authoring packages used for those assets for validation purposes. In this case it is common to export a LUT or other representation of the Look, much like animation and live-action. Color management must also be handled with defined import/export paths allowing management of externally created content, ideally via similar tools to VFX/Animation but are sometimes handled manually.

Key Points
Look Management visualises the Look through production and post-production
Irreversibly applying the Look is called “baking in the Look”
The Look should not be baked in until the final DI grading stage
Look Management should consider multiple gamut sources and deliverables
Metadata is crucial for good Look Management


