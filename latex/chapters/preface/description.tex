\section*{Description}
\unless\ifdefined\HCode
    \addcontentsline{toc}{section}{Description}
\fi

This updated paper presents an introduction to color management for motion-picture, television and games production, including visual effects, animation. It expands on the original’s focus on motion-picture visual effects and animation as a reflection of the expanded scope of digital imaging in the film and television production process, from capture to post-production to grading and final display, and as a reflection of the consistency in basic approaches used by visual effects, animation and increasingly games projects.

Color impacts many areas of the movie and the game-making process. From on-set capture to the many phases of the computer graphics pipeline: texture painting, look development, lighting, rendering, and compositing, to grading and finishing for the theater and home. Handling color is a tricky problem. This paper presents an introduction to color science, measuring and encoding color and core concepts like gamuts, transfer functions, and scene-referred and output-referred colorimetry. The paper then extends these concepts to their use in modern motion-picture and games production, including an introduction to efforts on digital color standardization in the motion-picture industry, ACES and ASC CDL, and how readers can experiment with all of these concepts for free using open-source software like OpenColorIO and Colour Science for Python.

