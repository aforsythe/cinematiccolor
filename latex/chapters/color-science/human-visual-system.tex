\section{Human Visual System}
\label{sec:human-visual-system}

Visual perception, a process by which humans acquire knowledge about their environment, is initiated when environmental light enters the eye and induces electrical signals subsequently processed within the brain where an image is formed. Acquiring knowledge is a cognitive process distinct from strict optical mechanisms. There are optical similarities between a camera and the eye in the way they capture an image of the environment but a camera does not have perceptual capabilities or cognitive abilities; it does not know about the world.

The principal components on the visual pathway are as follows:
Eye
Optic nerve
Optic chiasma
Optic tract
Lateral geniculate body
Optic radiation
Visual cortex
Visual association cortex

Principal components of the human visual system.
Wikipedia. (n.d.). Human visual pathway. Retrieved October 14, 2018, from https://commons.wikimedia.org/wiki/File:Human_visual_pathway.svg

Their study is beyond the scope of this document but it is important to highlight that while the light from the environment is captured and converted to electrical impulses by the eye, most of the information extraction happens in the cerebral cortex and multiple interpretations of the sensory stimulation are possible. 

Rotating snakes: Circular snakes appear to rotate spontaneously.
Kitaoka, A. (2003). Rotating snakes. Retrieved October 14, 2018, from http://www.ritsumei.ac.jp/~akitaoka/rotsnake.gif

Optical illusions are the result of light stimuli producing ambiguous and conflicting interpretations. They are important because they demonstrate that the stimulation of the eye does not entirely determine perception. Our perceptions correspond to the models the HVS has constructed rather than the original sensory stimulation. Palmer (1999) explains that "the observer constructs a model of what environmental situation might have produced the observed pattern of sensory stimulation." 
Key Points
Visual perception allows human to acquire knowledge about their environment.
Knowledge acquisition is a cognitive process.
Camera and the eye have optical similarities, but a camera does not have cognitive abilities.
Most of the information extraction happens in the cerebral cortex.
Optical illusions demonstrate that eye stimulation is not solely responsible for our perceptions: the HVS constructs a best-fitting model of the environmental situation.

\subsection{The Eye}
\label{subsec:the-eye}

The eye is of particular interest being the sensor with which the HVS probes the environmental light.

Cross section of the human eye.
https://commons.wikimedia.org/wiki/File:Eyesection.svg

The light pathway through the eye starts at the cornea, a transparent and curved organized group of tissue layers. The cornea represents the largest index of refraction (1.376) change at its interface with air. It contributes 3/4 of the eye focusing power although its focus is fixed.
Upon exiting the cornea, the light traverses the aqueous humor, a water-like fluid with a refractive index of 1.336, filling the anterior and posterior chambers of the eye and providing nutrition to the surrounding tissues.

Similar to a camera diaphragm, the iris, a ciliary muscle, controls the pupil size and therefore the amount of light reaching the retina. The pupil diameter varies from 7mm in dark viewing conditions to 3mm in bright viewing conditions

The light not absorbed by the iris enters the lens which provides the accommodation function (optical power change) by adjusting its shape allowing to focus at various distances. Its biconvex shape becomes flattened thus decreasing optical power to focus at a distance and fatter to increase optical power to focus at nearby objects. Its refraction index varies from edges to center (1.386 to 1.406) to reduce chromatic aberration.

The light traverses the vitreous humor, a thick fluid with a refractive index of 1.336, filling the space between the lens and the retina, to finally reach the retina.

Key Points
The eye is the HVS light sensor and is analogous to a camera.
Light enters the HVS through the cornea, the eye element with the strongest focusing power.
The iris controls the amount of light reaching the back of the eye by changing the pupil diameter akin to a camera diaphragm.
The lens allows the eye to focus on objects at varying distances by changing its shape, i.e. accommodation.

\subsection{Retina}
\label{subsec:retina}

The retina is a light-sensitive tissue composed of layers of neurons connected by synapses and receiving the optical image formed by the front eye elements. It contains the photoreceptor cells, the initial signal processing and transmission elements of the HVS.

We move our head and eyes so that the image of objects we look at falls on the fovea, a 1.5mm central pit area of the retina with increased density of photoreceptor cells and responsible for high-resolution vision.

Light traverses almost all the retinal layers before reaching the photoreceptor cells. Light triggers chemical changes in the photoreceptor cells which in turn, send a signal to the bipolar and horizontal cells. The signal is then propagated to the amacrine cells and ganglion cells, and finally to the optic nerve. 


Each synapse between the neural cells can perform an arithmetical operation such as amplification, gain control or nonlinear mapping, giving the eye the ability to perform spatial and temporal optical image sharpening.


The retina layers.
https://commons.wikimedia.org/wiki/File:Retina_layers.svg - colour-science.org

Key Points
The retina contains the photoreceptor cells, the initial signal processing and transmission elements of the HVS.
The fovea is a central retinal area with increased photoreceptor cells density, our heads and eyes are moving so that objects images fall on it.
Neural cells perform optical image sharpening.

\subsection{Photoreceptors}
\label{subsec:photoreceptors}

The photoreceptors are a type of neuron specialized for phototransduction, a process by which light is converted into electrical signals. There are two main classes of retinal photoreceptors:
Cone cell
Rod cells
The third class of photoreceptor cells within the retina is the Intrinsically Photosensitive Retinal Ganglion Cells (ipRGC) which play a significant role in the modulation of circadian rhythms, pupillary response, and adaptation.

\subsubsection{Cones and Photopic Vision}
\label{subsubsec:cones-and-photopic-vision}

Cone cells mediate photopic vision, which is vision under daytime illumination conditions, and are responsible for color perception. In the photoreceptor layer of the retina, cone cells of types L, M, S (sensitive to Long, Medium and Short wavelengths respectively) measure light with respective peak absorption at wavelengths of about 564 nm, 534 nm, and 420 nm.

Photopic vision luminance levels are usually defined for Luminance > 10 cd/m2.

The long, medium, and short cone fundamentals for 2 degrees showing respective peak absorption at wavelengths of about 564 nm, 534 nm, and 420 nm.

Most humans possess L, M, and S cone cells with similar distributions and peak absorptions, however a significant part of the population is affected by some form of color vision deficiency or color blindness. The most common types are:
Protanomaly: Defective L cone cells; the complete absence of L cone cells is known as Protanopia or red-dichromacy.
Deuteranomaly: Defective M cone cells with peak of sensitivity moved towards the red sensitive cones; the complete absence of M cone cells is known as Deuteranopia.
Tritanomaly: defective S cone cells, an alleviated form of blue-yellow color blindness; the complete absence of S cone cells is known as Tritanopia.
Monochromats only carry a single type of cone cells.
Tetrachromats carry four types of cone cells.

\subsubsection{Rods and Scotopic Vision}
\label{subsubsec:rods-and-scotopic-vision}

The rod cells mediate scotopic vision, which is vision under dark illumination conditions and where rods are the principle active photoreceptors. They measure light with peak absorption at a wavelength of about 507 nm.

Scotopic vision luminance levels are defined for Luminance < 0.001 cd/m2. The rod cells are about 100 times more sensitive to light than the cone cells.

\subsubsection{Mesopic Vision}
\label{subsubsec:mesopic-vision}
Mesopic vision is the result of the photopic vision and scotopic vision being active at the same time and occurs at low illumination conditions, and is usually defined for luminance in range 0.001 to 3 cd/m2.
CIE 1924 Photopic and CIE 1951 Scotopic Standard Observers. They are presented along with a Photopic Luminance Mesopic Luminous Efficiency Function modeling the sensitivity of the HVS for illumination levels between Photopic and Scotopic visions at 20% Photopic Luminance.

\subsubsection{Distribution}
\label{subsubsec:distribution}

The distribution of photoreceptors in the retina, note their total absence in the blind spot.
https://commons.wikimedia.org/wiki/File:Human_photoreceptor_distribution.svg

There are around 6.8×106 cones cells and approximately 110-125×106 rod cells in the retina. Cone cells are concentrated in the fovea and sparsely distributed in the peripheral retina. The asymmetry in the L, M and S cone cells distribution, the absence of S cone cells in the fovea centralis and their wide spacing into its periphery account for chromatic aberration. When the image is focused on the fovea centralis where reside the L and M cone cells, the S cone cells receive the shorter wavelength components. The axial chromatic aberration of the lens blur those components, thus a lower spatial resolution is required for the S cone cells.

No rod cells are located in the central fovea region, allowing for increased spatial acuity to be conveyed by the cone cells. The blind spot is a notable area without photoreceptors and where the ganglion cell axons leave the eye to form the optic nerve.

A single cone cell feeds its signals into a single ganglion cell while hundreds of rods pool their responses to feed into a single ganglion cell establishing a system of information compression between the photoreceptors and the optical nerve.

Key Points
There are two main classes of photoreceptor cells in the retina: cones and rods.
Cone cells are responsible for color perception, and photopic vision, the vision under daytime illumination conditions, with luminance levels defined for Luminance > 10 cd/m2.
Rod cells are responsible for scotopic vision, the vision under dark illumination conditions, with  luminance levels defined for Luminance < 0.001 cd/m2
Rod cells are about 100 times more sensitive to light than the cone cells.

\subsection{Dynamic Range}
\label{subsec:dynamic-range}

From starlight luminance levels around 10-4 cd/m2 to sunlight luminance levels reaching 105 cd/m2 or over 109 cd/m2 for a direct Sun luminance measurement, our world exhibits a wide dynamic range. In this context, dynamic range is the ratio between the maximum and minimum measurable light quantity in a scene.

The dynamic range of the human visual system.

The dynamic range of the HVS is the ratio between the most luminous stimulus causing complete photoreceptors bleaching with no damage and the smallest detectable light stimulus. Hood and Finkelstein (1986) and Ferwerda, Pattanaik, Shirley and Greenberg (1996) report from 10-6 to 10 cd/m2 for scotopic light levels and from 0.01 to 108 cd/m2 for the photopic range. However, the HVS, like any capture imaging devices, is unable to perceive this full range at once. Only a fraction is observable simultaneously, inducing HVS adaptation to illumination level variations.

The simultaneous dynamic range or steady-state dynamic range is defined as the ratio between the highest and lowest luminance values at which objects are detected while being in a state of full adaptation. It is the range of stimulus intensities over which photoreceptors are able to signal a change. Kunkel and Reinhard (2009) performed a series of psychophysical experiments on a high dynamic range display and determined that the HVS simultaneous dynamic range spans a range of 12.3 stops (3.7 log10 units) under illumination conditions with an adapting field varying from 1.78 cd/m2 to 17.8 cd/m2. They also found that the upper detection threshold was higher when the HVS is adapted to a brighter environment and thus that the maximum display luminance should be increased accordingly.

In the context of motion pictures and games, it is useful to talk about dynamic range in terms of photographic stops, i.e. doubling or halving of luminance. Stops are calculated as the log2 of the test luminance relative to a reference luminance level.

Relative exposure in stops is the log2 of the luminance relative to some reference exposure level. Any normalization factor suffices for relative comparisons.

EV
-8
-3
-2
-1
-0.5
0
0.5
1
2
3
8
2^EV
0.004
0.125
0.250
0.500
0.707
1
1.414
2
4
8
256
Scene-referred exposure values are often referenced in units of stops (EV), as the range between values is large for direct numerical comparisons. For example, it is difficult to get an intuition for what it means to reduce the exposure of an image by a factor of 0.004, it is generally more intuitive to refer to the same quantity as "-8 stops".

Key Points
The dynamic range of a scene is the ratio between the maximum and minimum measurable light quantity.
The dynamic range of the HVS is the ratio between the most intense non-damaging luminous stimulus and the smallest detectable light stimulus.
The HVS, like cameras, only perceive a fraction of the full extent of its dynamic range simultaneously.
The HVS simultaneous or steady-state dynamic range spans over 12.3 stops; the upper detection threshold rises with a brighter environment and thus requires brighter displays.
Dynamic range is often expressed in stops, a doubling or halving of luminance.

\subsection{Adaptation}
\label{subsec:adaptation}

The human visual system dynamically adapts to different illumination levels to improve the visual response. The three essential adaptation mechanisms of the HVS are:
Dark adaptation
Light adaptation
Chromatic adaptation

\subsubsection{Dark Adaptation}
\label{subsubsec:dark-adaptation}

Dark adaptation occurs when luminance level decreases. As a result of the lack of illumination, the observer’s visual system adapts to the stimuli and the visual sensitivity increases. Initially, upon entering a dark area, the cone cells’ sensitivity increases to reach full adaptation after around 10 minutes, when the rod cells’ sensitivity outperforms that of the cones, and complete adaptation is reached within 30 minutes.

A notable effect of the rod cells driving the visual system at low luminance level is that there is not enough light energy to be able to discriminate colors. Another notable effect,  the Purkinje Effect or Purkinje Shift, characterizes the HVS peak luminance sensitivity shift toward shorter wavelengths of the visible spectrum.

\subsubsection{Light Adaptation}
\label{subsubsec:light-adaptation}

Light adaptation is similar to dark adaption, but instead, the visual sensitivity decreases with luminance level increase. The adaptation process happening when entering a bright area is faster than dark adaptation. The rod cells first saturate as rhodopsin the photopigment of the rods, photo-bleaches, while the cone cells continue to adapt reaching peak sensitivity within 5-10 minutes.

\subsubsection{Chromatic Adaptation}
\label{subsubsec:chromatic-adaptation}

The CIE defines chromatic adaptation as the "visual process whereby approximate compensation is made for changes in the colo(u)rs of stimuli, especially in the case of changes in illuminants".

Chromatic adaptation controls the independent sensitivity of the three cones cells type and is the most important adaptation mechanism in color appearance. A white object viewed under different lighting conditions (daylight, tungsten or incandescent lighting) retains its white appearance because the sensitivity of the cone cells is independently adjusted to compensate for the changes in energy level at the wavelength ranges they are sensitive to. Chromatic adaptation can be thought of as analogous to the automatic white balancing feature of a camera.

It is important to make a distinction between the adapted white, defined by the CIE as the "colour stimulus that an observer who is adapted to the viewing environment would judge to be perfectly achromatic and to have a luminance factor of unity" and the adopted white defined as the "spectral radiance distribution as seen by an image capture or measurement device and converted to colour signals that are considered to be perfectly achromatic and to have an observer adaptive luminance factor of unity; i.e., colour signals that are considered to correspond to a perfect white diffuser". The adopted white, used in the color imaging system, is specified and known while the adapted white of the HVS can only be estimated.

Chromatic adaptation occurs at a faster rate than dark and light adaptation. Rinner and Gegenfurtner (2000) measured two separate adaptation mechanisms representing 40% of the total adaptation for color appearance and 100% of the total adaptation for color discrimination: a fast component with a half-life of 40–70ms and a slower component with a half-life of about 20s, agreeing with the findings of Fairchild and Reniff (1995). They also identified a third higher order component with a half-life under 10ms. Induced by cortical computations, it is situated after the other adaptive stages. It only affects color appearance and is responsible for 60% of the total adaptation for color appearance. Chromatic adaptation reaches 90% completion after 60s and is complete within 2 minutes

Through a series of experiments that measured the spatial, temporal, and chromatic properties of chromatic-adaptation mechanisms, Fairchild (1993) has shown clear evidence that along the well known sensory mechanisms supporting an automatic response to a stimulus, e.g. retinal gain control, there were also cognitive mechanisms dependent on the observer knowledge of the scene content.
The cognitive mechanisms are very effective when viewing hard-copy images, e.g. print of a photograph: they allow the observer to discount the scene illuminant, but they do not work with soft-copy displays, e.g. a monitor displaying the photograph, as those cannot be interpreted as illuminated objects and thus only sensory mechanisms are active.

Key Points
Lack of illumination triggers dark adaptation, rod cells sensitivity increases and outperforms that of the cones; full adaptation occurs within 30 minutes.
Light adaptation happens when entering a bright area, rod cells saturate while cone cells are adapting;  full adaptation occurs within 5-10 minutes.
Chromatic adaptation controls the independent sensitivity of the cone cells, causing objects to retain their appearance under different illumination conditions; chromatic adaptation is fast: it happens in the course of a few milliseconds and reaches completion within 2 minutes.
Chromatic adaptation is commonly compared to the white-balancing feature of a camera.

\subsection{Non-Linearity of the HVS}
\label{subsec:non-linearity-of-the-hvs}

The response of the rods to increasing field luminance on a log-log scale. Davson, H. (1990). Physiology of the Eye. Macmillan International Higher Education. ISBN:134909997X - colour-science.org

The just-noticeable difference (JND) is the minimum change in stimulus intensity required to produce a detectable variation in sensory experience. Weber's law states that the JND between two stimuli is proportional to the magnitude of the stimuli: an increment is judged relative to the previous amount.

In Elements of psychophysics, Fechner (1860) mathematically characterized Weber’s law showing that it follows a logarithmic transformation: the subjective sensation of a stimulus is proportional to the logarithm of the stimulus intensity. Fechner’s scaling has been found to apply to the perception of brightness, at moderate and high brightness, with perceived brightness being proportional to the logarithm of the actual intensity. 

At lower levels of brightness, the De Vries-Rose law applies which states that the perception of brightness is proportional to the square root of the actual intensity. 

Stevens generalises Fechner's law: the results of the physical-perceptual relationship of his experiments on a logarithmic scale were characterized by straight lines with different slopes, suggesting that the relationship between perceptual magnitude and stimulus intensity follows a power law with varying exponent.


Perceived magnitude of stimuli of increased intensity following a power law with varying exponent and displayed on a linear scale. Stevens, S. S. (1975). Psychophysics: introduction to its perceptual, neural, and social prospects. (Wiley, Ed.) (2nd ed.). Wiley. ISBN:9780471824374 

Stimuli of figure 2.x.x displayed on a log-log scale and characterized by straight lines with different slopes. Stevens, S. S. (1975). Psychophysics: introduction to its perceptual, neural, and social prospects. (Wiley, Ed.) (2nd ed.). Wiley. ISBN:9780471824374

Because of the various HVS adaptation mechanisms, perceived brightness has a non-linear relationship with the actual physical intensity of the stimulus. A cube root commonly approximates it. Multiple models of lightness were proposed leading to the creation of CIE L* in 1976. CIE L* characterizes the perceptual response to relative luminance.


CIE L* characterizes the perceptual response to relative luminance.

CIE L* was developed for colorimetric measurements of colored samples under a uniform illumination source. It was not tested for high illumination conditions with color stimuli orders of magnitude below or above a perfect white diffuse reflector. The resulting uncertainty in Lightness prediction for High Dynamic Range (HDR) imaging applications leads scientific research into searching for a better function. Fairchild and Wyble (2010) and Fairchild and Chen (2011) proposed a new physiologically-plausible hyperbolic function based on Michaelis-Menten kinetics, a model of enzyme kinetics. Abebe, Pouli, Larabi, and Reinhard (2017) modified Fairchild and Chen (2011) function to account for emissive color stimuli. As the writing of this document, the CIE has not yet adopted a new suitable function.

With the related objective of finding a function adapted to HDR image formation, Miller (2014) designed the Perceptual Quantizer (PQ). It is an important function, standardized by the Society of Motion Picture and Television Engineers as SMPTE ST 2084 and this document frequently refers to it.

Key Points

Writing in-progress…


