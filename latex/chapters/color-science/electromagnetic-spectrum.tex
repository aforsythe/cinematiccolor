\section{Electromagnetic Spectrum}%
\label{sec:electromagnetic-spectrum}

Color is a perceived sensation that occurs when light interacts with the HVS. The CIE defines light as the “electromagnetic radiation that is considered from the point of view of its ability to excite the human visual system.”

The electromagnetic spectrum from approximately 360-780 nm is visible to human observers and called the visible spectrum. Note that color sensations produced above are limited to those the particular display used to view this document can represent.

In Opticks, Isaac Newton (1704) investigates the fundamental nature of light and color by describing the refraction of light by prisms. Newton demonstrated that a prism can separate seemingly achromatic light into components perceived as distinct hues, with each hue refracted at a unique and consistent angle; that this separated light may be recombined, either entirely to recover the achromatic color, or selectively, by which he could produce any desired color sensation.

Comparison of an original scene on the left and a version where its light is dispersed horizontally with a Star Analyser 100 (SA-100) diffraction grating demonstrating that the seemingly white light from the Sun illuminating the scene is composed of distinct hues.

Newton asserted that color is a visual sensation and not an inherent property of the light or object materials. He supported this assertion by noting that a magenta (purple) sensation can be created by combining red and violet components of two spectra, but this sensation is not produced by any single component of refracted light.

The measurement of the radiation across the entire electromagnetic spectrum is known as radiometry. In this document, the focus is almost entirely on radiation within the visible spectrum. Photometry is a subset of radiometry concerned solely with radiation within the visible spectrum, where the measurement of energy at a particular wavelength in that spectrum is weighted by the sensitivity of the HVS at that wavelength. A comprehensive description of radiometry and photometry is beyond the scope of this document, but the next sections define some necessary radiometric and photometric quantities used in digital imaging and color science. Their descriptions below are taken from Wyszecki and Stiles (2000).

\subsection{Radiometric Quantities of Interest}%
\label{subsec:radiometric-quantities-of-interest}

Radiant power (or radiant flux) (Pe) is radiant energy emitted, transferred, or received through a surface, in a unit time interval. 

It has a unit of Watt (Joules.sec-1).
Irradiance (Ee) at a point of a surface is the quotient of the radiant power incident on an infinitesimal surface element containing the point, by the area of that surface element. It has a unit of Watt.m-2.
Radiance Exitance (or radiosity) (Me) at a point of a surface is the quotient of the radiant power emitted by an infinitesimal surface element containing the point, by the area of that surface element. 

It has a unit of Watt.m-2.
Radiance Intensity (Ie) of a source in a given direction is the quotient of the radiant power emitted by the source in an infinitesimal element of solid angle containing the given direction by the element of solid angle. 

It has a unit of Watt.sr-1.
Radiance (Le) in a given direction at a point on the surface of a source or a receiver, or at a point on the path of a beam, is the quotient of the radiant power leaving, arriving at, or passing through an element of surface at this point and propagated in directions defined by an elementary cone containing the given direction, by the product of the solid angle of the cone and the area of the orthogonal projection of the surface element on a plane perpendicular to the given direction. 

It has a unit of Watt.m-2.sr-1.

\subsection{Photometric Quantities of Interest}%
\label{subsec:photometric-quantities-of-interest}

Photometric quantities
Luminous power (or luminous flux) (Pv) is the quantity derived from radiant power by evaluating the radiant energy according to its action upon a selective receptor, the spectral sensitivity of which is defined by a standard luminous efficiency function. 

It has a unit of Lumen (lm.sec-1).
Illuminance (Ev) at a point of a surface is the quotient of the luminous power incident on an infinitesimal element of the surface containing the point under consideration, by the area of that surface element. 

It has a unit of lm.m-2 (or Lux).
Luminous Exitance (or luminous emittance) (Mv) from a point on a surface is the quotient of the luminous power emitted from an infinitesimal element of the surface containing the point under consideration, by the area of that surface element. 

It has a unit of lm.m-2 (or Lux).
Luminous Intensity (Iv) in a given direction is the quotient of the luminous power emitted by a point source in an infinitesimal cone containing the given direction, by the solid angle of that cone. 

It has a unit of candela (cd).
Luminance (Lv) at a point of a surface and in a given direction is the quotient of the luminous intensity in the given direction of an infinitesimal element of the surface containing the point under consideration, by the orthogonally projected area of the surface element on a plane perpendicular to the given direction. 

It has a unit of \si{\candela\per\meter\squared}. A convenient non-SI (International System of Units) shorthand for the luminance unit is nit from the latin nitere, to shine.
Relative Luminance (Y) is luminance normalized to the range 0-1 or 0-100, relative to a reference white Luminance value.
Key Points
Color is perceived when light interacts with the HVS.
Light is the visible portion of the electromagnetic spectrum.
Achromatic light can be separated into distinct hues by prisms. Those distinct hues can be recombined by prisms to recover the achromatic light.
Color is not an inherent property of the light or object materials:  magenta does not exist as a distinct hue but is perceived when combined red and blue hues.
Radiometry is the measurement of radiation across the entire electromagnetic spectrum.
Photometry is the measurement of radiation weighted by the sensitivity of the HVS across the visible spectrum.

\subsection{Spectral Distribution}%
\label{subsec:spectral-distribution}

In radiometry and photometry, the radiant power emitted by a light source or illuminant, or reflected, transmitted or absorbed by a surface is characterized by a spectral distribution giving the power of the light per unit area per unit wavelength or the percentage of light reflected, transmitted or absorbed per unit wavelength.


Four CIE illuminants spectral power distributions. A and D65 are respectively CIE Standard Illuminant A and CIE Standard Illuminant D65 and are described in section 2.3.4. F2 is CIE Illuminant F2, a fluorescent type light and HeNe Laser (Normalised) is a Helium-Neon laser measured by Deglr6328 (2006).

The visible spectrum is considered for electromagnetic radiation with wavelengths in the range of approximately 360-780 nanometers (nm). The spectral sensitivity of the human eye peaks under daylight-like illumination towards the middle of this range, at 555nm (yellow-green). At the extremes, radiant energy below 360nm (ultraviolet)  or above 780nm (infrared) appears indistinguishable from black, no matter how intense.


Reflectance, transmittance and absorptance spectra of a Sorghum Halepense leaf from LOPEX93. Chlorophyll absorbs a large quantity of blue and red radiation giving leaves their green color.
Light, through its fluctuations and spectral quality, has contributed to shaping the adaptation of many species. Chlorophyll, contained in plants chloroplasts, is a pigment which absorbs light during photosynthesis. It absorbs violet, blue and red wavelengths while reflecting green wavelengths.

Sparks, DasSarma, and Reid (2006) and DasSarma and Schwieterman (2018) have proposed the hypothesis that early photosynthetic organisms may have used retinal pigment which has a peak absorption centered at 550nm, thus absorbing most of the green wavelengths. The chlorophyll based organisms that appeared later adapted by absorbing the remaining regions of the electromagnetic spectrum.



Solar spectral irradiance compared to CIE 1924 Photopic Standard Observer, note how both peak around 555nm.

The solar spectral irradiance is thought to be a fundamental driver of the evolution of the HVS and peaks in the central region of the visible spectrum. Boynton (1990) says that "this coincidence is probably not accidental, but is more likely the product of biological evolution." Wang, Tang, and Yan (2011) have shown that the spectral sensitivities of photoreceptors of different species of moray eel were correlated with the photic characteristics of their habitats.

Key Points
A spectral distribution characterizes, at each wavelength, the radiant power emitted by a light source or the percentage of light reflected, transmitted or absorbed by a surface.
The wavelengths range of the visible spectrum is approximately \SIrange{360}{780}{\nm}.
Spectral sensitivity of the HVS peaks at 555nm, ultraviolet radiation below 360nm or infrared radiation above 780nm is invisible.
The solar spectral irradiance peaks in the central region of the visible spectrum and is thought to be an essential driver of the evolution of the HVS.


