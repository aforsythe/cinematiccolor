\section{Grassmann's Law of Additive Color Mixture}%
\label{sec:grassmann-laws-of-additive-color-mixture}

There are different formulations of the Grassmann's law of additive color mixture in the litterature, the following formulation is given by Wyszecki and Styles (2000):
Symmetry Law
If color stimulus A matches color stimulus B, then color stimulus B matches color stimulus A.
Transitivity Law
If A matches B and B matches C, then A matches C.
Proportionality Law
If A matches B, then αA matches αB, where α is any positive factor by which the radiant power of the color stimulus is increased or reduced, while its relative spectral distribution is kept the same.
Additivity Law
If A, B, C, D are any four color stimuli, then if any of two of the following three conceivable color matches A matches B, C matches D, and (A + C) matches (B + D) holds good, then so does the remaining match (A + D) matches (B + C) where (A + C), (B + D), (A + D), (B + C) denote, respectively additive mixtures of A and C, B and D, A and D and B and C.
