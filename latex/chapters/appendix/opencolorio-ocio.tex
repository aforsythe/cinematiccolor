\section{OpenColorIO (OCIO)}
\label{sec:opencolorio-ocio}

OpenColorIO (OCIO) is an open-source color pipeline originally created by Jeremy Selan, and sponsored by Sony Picture Imageworks. It is currently maintained by a group of developers including X Y Z from companies A B C. OpenColorIO has two major goals: consistent color transforms, and consistent image display, across multi-application cinematic color pipelines.

On the left is a log DPX image loaded in a compositing application. On the right, a scene-referred linear EXR representation of the same image is loaded in a different application. Both applications utilize OpenColorIO to provide matched image display and color space conversions, referencing an externally defined color configuration. Image courtesy of the Kodak Corporation.
The design goal behind OCIO is to decouple the color pipeline API from the specific color transforms selected, allowing supervisors to tightly manage and experiment with color pipelines from a single location. Unlike other color management solutions such as ICC, OpenColorIO is natively designed to handle both scene-referred and display-referred imagery. All color transforms are loaded at runtime, from a color configuration external to any individual application. OCIO does not make any assumptions about the imagery; all color transformations are “opt-in.” This is different from the color management often built-in to applications, where it is often difficult to track down the specific LUTs / gamma transforms automatically applied without the user’s awareness.

OCIO color configuration files define all of the conversions that may be used. For example, if you are using a particular camera’s color space, one would define the conversion from the camera’s color encoding to scene-referred linear. You can also specify the display transforms for multiple displays in a similar manner. OCIO transforms can rely on a variety of built-in building-blocks, including all common math operations and the majority of common lookup table formats. OCIO also has full support for both CPU and GPU pathways, in addition to full support for CDLs and per-shot looks.

The OCIO project also includes some of the real color configurations for film productions, such as those used in Cloudy with a Chance of Meatballs and Spider-Man, enabling users to experiment with validated color pipelines. OCIO also ships with a configuration compatible with Academy ACES 1.0 and greater, allowing the use of an ACES pipeline in applications without native support.

OpenColorIO is in use at many of the major visual effects and animation studios, and is also supported out of the box in a variety of commercial software. See opencolorio.org for up-to-date information on supported software, and to download the source code for use at home.

